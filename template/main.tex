\documentclass{article}
\usepackage[utf8]{inputenc}
\usepackage{fullpage}
\usepackage[russian]{babel}
\usepackage{amssymb}
\usepackage{listings}
\usepackage{xcolor}
\usepackage{hyperref}
\usepackage{amsthm,amsmath,amsfonts,amssymb}
\usepackage{graphicx}

\lstset{
    language=Python,
    commentstyle=\color{gray!80!white}
}
\hypersetup{
    colorlinks=true,
    urlcolor=blue
}

\newcommand{\ProbSimple}[1]{\mathbf{#1}}
\newcommand{\ProbSign}[3]{\underset{#2}{\ProbSimple{#1}}\left[\;#3\;\right]}
\renewcommand{\Pr}[2]{\ProbSign{Pr}{#1}{#2}}
\newcommand{\Prw}[1]{\Pr{}{#1}}
\newcommand{\Ex}[2]{\ProbSign{E}{#1}{#2}}
\newcommand{\Exw}[1]{\Ex{}{#1}}
\newcommand{\Dp}[2]{\ProbSign{D}{#1}{#2}}
\newcommand{\Dpw}[1]{\Dp{}{#1}}
\newcommand{\F}{\mathbb{F}}
\newcommand{\Prr}{\ProbSimple{Pr}}
\newcommand{\solut}[2]{\textit{\paragraph{Задача.} #1}\vspace{0.2cm}\textbf{Решение. }#2\vspace{0.2cm}}


\begin{document}

\begin{center}
    {\Large Решения задач по курсу ``Алгоритмы обработки потоковых данных''}
    \vspace{0.3cm}

    {\large Казань, 2016}
\end{center}
\begin{flushright}
\color{red}{\textbf{Василий Пупкин}}\par
\end{flushright}

\solut{Докажите, что для любых $a, b \geq 0$ справедливо неравенство
$$
\frac{a + b}{2} \geq \sqrt{a \cdot b}.
$$}{
Возведем обе части неравенства в квадрат и получим:
$$
\frac{(a + b)^2}{4} \geq a \cdot b.
$$
Далее
$$
a^2 + 2 \cdot a \cdot b + b^2 \geq 4 \cdot a \cdot b,
$$
или
$$
a^2 - 2 \cdot a \cdot b + b^2 \geq 0,
$$
или 
$$
(a - b)^2 \geq 0.
$$

Отметим также, что 
$$
\Prw{\frac{a + b}{2} \geq \sqrt{a \cdot b}} = 1
$$
и
$$
\begin{array}{rl}
\Exw{2 + 2} & = 4\\
\Dpw{2 + 2} & = 0
\end{array}
$$
}

\end{document}